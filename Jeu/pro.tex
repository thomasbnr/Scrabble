\pdfminorversion 7
\pdfobjcompresslevel 3

\documentclass[a4paper]{article}

\usepackage[utf8]{inputenc}
\usepackage[T1]{fontenc}
\usepackage[francais]{babel}
\usepackage[bookmarks=false,colorlinks,linkcolor=yellow]{hyperref}
\usepackage[top=1.5cm,bottom=1.5cm,left=1.5cm,right=1.5cm]{geometry}
\usepackage{verbatim}
\usepackage{xcolor}
\hypersetup{
  pdftitle={Projet 2019-2020 LaTeX},
  pdfsubject={Scrabble},
  pdfkeywords={Imperative},
  pdfauthor={Bonnerot Thomas Rouichi Adil }
}

\begin{document}

\begin{center}\section*{Rapport}
{\Large\bf
Programmation Impérative 2\\}
Projet 2019-2020~: \\
\textbf{Scrabble}
\end{center}

Nous avons décidé ensemble de prendre le Scrabble comme projet. C'est un jeu connu par beaucoup de personnes, cela nous intéressait et nous voulions rester dans un concept de réalisation de jeu.

\section{Plateau}


Premièrement, nous avons regardé les règles pour avoir une idée de l'organisation de notre plateau. En effet, chaque cases de notre plateau possèdent une valeur différente. De plus chaque lettres possèdent un score différent et une occurence différente, c'est à dire que plus la lettre peut être utilisé dans un mot, plus il y'a de chances de la retrouver dans notre main. Par ailleurs, il a fallut penser au joker, qui peut prendre comme valeur n'importe quelle lettre.
Initialisation du plateau:
\begin{verbatim}
  
A mettre 


\end{verbatim}



\section{Tris}

Voici quelques fonctions permettant de trier les mots qui sont possibles par rapport au dictionnaire officiel du scrabble,en fonction des groupes de mots disponibles sur le plateau. Nous avons eu pas mal de problèmes dans l'optimisation de nos différents tries.
\\

\begin{verbatim}


A mettre 
\end{verbatim}


\section{Structures}

Ici notre fichier prototype contenant l'ensemble de nos strctures et fonctions ;
\begin{verbatim}

A mettre 

\end{verbatim}

\section{Les parcours}

Voici l'algorithme des différents modes de parcours sur le plateau, de manière horizontale ou verticale et en fonction des différents groupes de mots qui sont disponibles sur le plateau. Il cherche à effectué le meilleur coup possible.

\begin{verbatim}

A mettre §

\end{verbatim}


\section{main}

Voici notre main  : 

\begin{verbatim}

A mettre
\end{verbatim}

\section{Makefile}


\begin{verbatim}
CC = gcc
CFLAGS = -Wall
EXEC=scrabble

all: $(EXEC)

scrabble: fonctions.o main.o
  $(CC) -o scrabble fonctions.o main.o

main.o: main.c scrabble.h
    $(CC) -o main.o -c main.c $(CFLAGS)


fonctions.o:  fonctions.c scrabble.h
        $(CC) -o fonctions.o -c fonctions.c $(CFLAGS)


clear:  
    rm *.o *~ core

\end{verbatim}


\color{red}
\section{Conclusion}

Ce projet nous a permis de renforcer notre niveau en C. Malgré cela nous avons rencontré pas mal de difficultés, notamment dans les allocations de mémoires et l'utilisation de pointeurs. Par exemple lorsqu'il fallait modifier des variables sans passer par le main. Neanmoins grâce à une bonne répartion des taches, une bonne organisation et à des nuits écourtées, nous avons pû finir le projet à temps.



\end{document}